% !TeX root = ./main.tex
\documentclass[main]{subfiles}
\begin{document}
\chapter{Теорема Коши}
\section{Теорема Коши для прямоугольника}
\begin{theorem*}[Теорема 6.4 из прошлого семестра]
    \begin{gather*}
        f \in C([a,b] \times [p,q]) \\
        g(y) = \int^b_a f(x,y) dx \quad h(x) = \int^q_p f(x,y) dy
        \intertext{По теореме о непрерывности интеграла, зависящего от параметра $g \in C([p,q])$ и $h \in C([a,b])$. Тогда}
        \int^b_a h(x) dx = \int^q_p g(y) dy
    \end{gather*}
\end{theorem*}
\begin{theorem} \label{2:intOfInt}
    Формула интегрирования интеграла, зависящего от параметра в теореме выше справедлива для комплекснозначных функций $f$.
\end{theorem}
\begin{proof}
    Следует из применения формулы к вещественной и мнимой частям функции $f$.
\end{proof}

\begin{theorem}\label{2:cauchy_square}
    Пусть $Q$~--- прямоугольник.
    \[Q = \{z = x + iy: a \le x \le b, p \le y \le q\}\]
    $G$~--- открытое множество, $Q \subset G$, $f \in A(G)$.
    Тогда
    \[\AnClOrInt[\AnClOrCurve[\partial Q]] f(z) dz = 0, \tag{7}\]
    где $\AnClOrCurve[\partial Q]$~--- ориентированная граница $Q$.
\end{theorem}
\begin{longProof}
    Не уменьшая общности, считаем, что $\AnClOrCurve[\partial Q]$ ориентирована положительно.
    Пусть $A = a + ip$, $B = b + ip$, $C = b + iq$, $D = a + iq$.
    \begin{center}
        \import{figures}{cauchy_theorem_square.pdf_tex}
    \end{center}
    $\AnClOrCurve[AB]$~--- ориентированная кривая, являющаяся отрезком $[A, B]$ с началом $A$ и концом $B$, аналогично $\AnClOrCurve[BC]$, $\AnClOrCurve[CD]$, $\AnClOrCurve[DA]$, тогда
    \[\AnClOrInt[\AnClOrCurve[\partial Q]] \dotsi = \AnClOrInt[\AnClOrCurve[AB]] \dotsi + \AnClOrInt[\AnClOrCurve[BC]] \dotsi + \AnClOrInt[\AnClOrCurve[CD]] \dotsi + \AnClOrInt[\AnClOrCurve[DA]] \dotsi,\]
    пользуясь параметризацией отрезков:
    \begin{gather*}
        \AnClOrCurve[AB] = \{z = x + ip: x \in [a,b]\} \\
        \AnClOrCurve[BC] = \{z = b + iy: y \in [p,q]\} \\
        \AnClOrCurve[CD] = \{z = t + iq: t = a + b - x, x \in [a,b]\}\\
        \AnClOrCurve[DA] = \{z = a + is: s = p + q - y, y \in [p,q]\}
    \end{gather*}
    Пусть $\ClOrCurve[AD]$~--- отрезок, противоположно ориентированный по отношению к $\AnClOrCurve[AD]$, $\ClOrCurve[CD]$~--- отрезок, противоположно ориентированный по отношению к $\AnClOrCurve[CD]$.
    Тогда имеем
    \[ \AnClOrInt[\AnClOrCurve[\partial Q]] f(z) dz = \left( \AnClOrInt[\mathrlap{\AnClOrCurve[AB]}] f(z) dz - \ClOrInt[\mathrlap{\ClOrCurve[CD]}] f(z) dz \right) + \left( \AnClOrInt[\mathrlap{\AnClOrCurve[BC]}] f(z) dz - \ClOrInt[\mathrlap{\ClOrCurve[AD]}] f(z) dz \right) \tag{8} \]
    \begin{multline*}
        \AnClOrInt[\mathrlap{\AnClOrCurve[AB]}] f(z) dz - \ClOrInt[\mathrlap{\ClOrCurve[CD]}] f(z) dz = \int_{a}^{b} f(x + ip) dx - \int_{a}^{b} f(x + iq) dx = \\
        = - \int_{a}^{b} \left(f(x + iq) - f(x + ip)\right) dx = - \int_{a}^{b} \left(f^*(x, q) - f^*(x, p)\right) dx = \\
        = - \int_{a}^{b} \left( \int_{p}^{q} {f^*}_y' (x, y) dy\right) dx \tag{9}
    \end{multline*}
    В предыдущем интеграле в $(9)$ мы воспользовались при фиксированном $x$ формулой Ньютона-Лейбница, что возможно, поскольку $f^* \in C^1(G)$, что влечет $f^* \in C^1(Q)$.
    Далее,
    \[\AnClOrInt[\mathrlap{\AnClOrCurve[BC]}] f(z) dz - \ClOrInt[\mathrlap{\ClOrCurve[AD]}] f(z) dz = i \int_{p}^{q} f(b + iy) dy - i \int_{p}^{q} f(a + iy) dy \tag{10}\]
    $(10) \implies$
    \begin{multline*}
        \AnClOrInt[\mathrlap{\AnClOrCurve[BC]}] f(z) dz - \ClOrInt[\mathrlap{\ClOrCurve[AD]}] f(z) dz = i \int_{p}^{q} \left( f(b + iy) dy - f(a + iy) \right) dy = \\
        = i \int_{p}^{q} (f^*(b, y) - f^*(a, y)) dy = i \int_{p}^{q} \left( \int_{a}^{b} {f^*}_x' (x, y) dx\right) dy \tag{11}
    \end{multline*}
    В $(11)$ мы опять воспользовались формулой Ньютона-Лейбница.
    Применяя теорему \ref{2:intOfInt}, из соотношений $(8)$--$(11)$ находим, что
    \begin{multline*}
        \AnClOrInt[\AnClOrCurve[\partial Q]] f(z) dz = i \int_{p}^{q} \left( \int_{a}^{b} {f^*}_x' (x, y) dx\right) dy - \int_{a}^{b} \left( \int_{p}^{q} {f^*}_y' (x, y) dy\right) dx = \\
        = - \int_{a}^{b} \left( \int_{p}^{q} {f^*}_y' (x, y) dy\right) dx + i \int_{a}^{b} \left( \int_{p}^{q} {f^*}_x' (x, y) dy\right) dx = \\
        = \int_{a}^{b} \left( \int_{p}^{q} \left( i {f^*}_x' (x, y) - {f^*}_y'(x, y)\right) dy \right) dx = \\
        = i \int_{a}^{b} \left( \int_{p}^{q} \left( {f^*}_x' (x, y) + i {f^*}_y'(x, y)\right) dy \right) dx = \\
        = 2 i \int_{a}^{b} \left( \int_{p}^{q} \frac{1}{2} \left( {f^*}_x' (x, y) + i {f^*}_y'(x, y)\right) dy \right) dx = \\
        = 2 i \int_{a}^{b} \left( f_{\overline{z}}' (x + iy) dy\right) dx = 0,
    \end{multline*}
    поскольку $f \in A(G)$.
\end{longProof}

\section{Теорема Коши для прямоугольного треугольника}
\begin{theorem}
    Пусть $A = a + ip$, $B = b + ip$, $C = b + iq$, $a < b$, $p < q$, $\Delta \subset \C$~--- треугольник с вершинами $A,B,C$, $G$~--- открытое множество, $\Delta \subset G$, $f \in A(G)$.
    Тогда
    \[\AnClOrInt[\AnClOrCurve[\partial \Delta]] f(z) dz = 0, \tag{12}\]
    где $\AnClOrCurve[\partial \Delta]$~--- ориентировання граница $\Delta$.
\end{theorem}
\begin{longProof}
    Пусть
    \begin{align*}
        D & = \frac{a + b}{2} + i \frac{p + q}{2} & A_1 & = \frac{a + b}{2} + ip & C_1 & = b + i \frac{p + q}{2}
    \end{align*}
    \begin{center}
        \import{figures}{cauchy_theorem_triangle.pdf_tex}
    \end{center}
    Тогда
    \begin{gather*}
        \AnClOrInt[\AnClOrCurve[C_1 D]] \dotsi + \AnClOrInt[\AnClOrCurve[D C_1]] \dotsi = 0 \\
        \AnClOrInt[\AnClOrCurve[A_1 D]] \dotsi + \AnClOrInt[\AnClOrCurve[D A_1]] \dotsi = 0
    \end{gather*}
    \begin{multline*}
        \AnClOrInt[\AnClOrCurve[\partial \Delta]] \dotsi = \AnClOrInt[\AnClOrCurve[AB]] \dotsi + \AnClOrInt[\AnClOrCurve[BC]] \dotsi + \AnClOrInt[\AnClOrCurve[CA]] \dotsi = \\
        = \AnClOrInt[\AnClOrCurve[AA_1]] \dotsi + \AnClOrInt[\AnClOrCurve[A_1 B]] \dotsi + \AnClOrInt[\AnClOrCurve[BC_1]] \dotsi + \AnClOrInt[\AnClOrCurve[C_1 C]] \dotsi + \AnClOrInt[\AnClOrCurve[CD]] \dotsi + \AnClOrInt[\AnClOrCurve[DA]] \dotsi = \\
        = \left( \AnClOrInt[\AnClOrCurve[AA_1]] \dotsi + \AnClOrInt[\AnClOrCurve[A_1 D]] \dotsi + \AnClOrInt[\AnClOrCurve[DA]] \dotsi \right) + \\
        + \left( \AnClOrInt[\AnClOrCurve[C_1 C]] \dotsi + \AnClOrInt[\AnClOrCurve[CD]] \dotsi + \AnClOrInt[\AnClOrCurve[DC_1]] \dotsi\right) + \left( \AnClOrInt[\AnClOrCurve[DA_1]] \dotsi + \AnClOrInt[\AnClOrCurve[A_1 B]] \dotsi + \AnClOrInt[\AnClOrCurve[BC_1]] \dotsi + \AnClOrInt[\AnClOrCurve[C_1 D]] \dotsi\right) = \\
        = \AnClOrInt[\AnClOrCurve[\partial \Delta_1]] \dotsi + \AnClOrInt[\AnClOrCurve[\partial \Delta_2]] \dotsi + \AnClOrInt[\AnClOrCurve[\partial Q_1]] \dotsi, \tag{13}
    \end{multline*}
    где $\Delta_1$~--- треугольник с вершинами $A, A_1, D$, $\Delta_2$~--- треугольник с вершинами $C_1, C, D$, $Q_1$~--- прямоугольник с вершинами $A_1, B, C_1, D$.
    По теореме \ref{2:cauchy_square} имеем равенство
    \[\AnClOrInt[\AnClOrCurve[\partial Q_1]] f(z) dz = 0, \tag{14}\]
    поэтому $(13)$ и $(14) \implies$
    \[ \AnClOrInt[\AnClOrCurve[\partial \Delta]] f(z) dz = \AnClOrInt[\AnClOrCurve[\partial \Delta_1]] f(z) dz + \AnClOrInt[\AnClOrCurve[\partial \Delta_2]] f(z) dz\]
    Приведенное рассуждение можно применить к $\Delta_1$ и к $\Delta_2$ и т.д., в результате, если поделить гипотенузу $AC$ на $2^n$ равных отрезков и построить подобные $ABC$ треугольники $\Delta_{n1}, \dotsc, \Delta_{n2^n}$, занумерованные снизу вверх, то получим равенство

    \begin{minipage}{0.3\textwidth}
        \begin{center}
            \import{figures}{cauchy_theorem_triangle_ntwo.pdf_tex}
        \end{center}
    \end{minipage}
    \begin{minipage}{0.6\textwidth}
        \[\AnClOrInt[\AnClOrCurve[\partial \Delta]] f(z) dz = \sum_{k=1}^{2^n} \AnClOrInt[\AnClOrCurve[\partial \Delta_{nk}]] f(z) dz \tag{15}\]
    \end{minipage}

    По теореме Кантора функция $f$ равномерно непрерывна в $\Delta$, поэтому
    \begin{multline*}
        \forall \epsilon > 0\ \exists \delta > 0 \text{ т.ч. } \forall z_1, z_2 \in \Delta \text{ т.ч. } |z_2 - z_1| < \delta \\
        \text{выполнено } |f(z_2) - f(z_1)| < \epsilon.
    \end{multline*}
    \begin{minipage}{0.45\textwidth}
        \begin{center}
            \def\svgwidth{0.5\textwidth}
            \import{figures}{cauchy_theorem_triangle_abg.pdf_tex}
        \end{center}
    \end{minipage}
    \begin{minipage}{0.45\textwidth}
        Выберем $N$ так, чтобы $2^{-N} |C - A| < \delta$, и возьмем $n > N$.
        Пусть $\alpha, \beta, \gamma$~--- вершины треугольника $\Delta_{nk}$.
    \end{minipage}

    Тогда
    \begin{multline*}
        \AnClOrInt[\AnClOrCurve[\partial \Delta_{nk}]] f(z) dz = \AnClOrInt[\AnClOrCurve[\partial \Delta_{nk}]] f(\alpha) dz + \AnClOrInt[\AnClOrCurve[\partial \Delta_{nk}]] (f(z) - f(\alpha)) dz =\\
        = f(\alpha) \AnClOrInt[\mathrlap{\AnClOrCurve[\partial \Delta_{nk}]}] 1 dz + \AnClOrInt[\AnClOrCurve[\partial \Delta_{nk}]] (f(z) - f(\alpha)) dz = 0 + \AnClOrInt[\AnClOrCurve[\partial \Delta_{nk}]] (f(z) - f(\alpha)) dz, \tag{16}
    \end{multline*}
    далее, с учетом $|z - \alpha| \le |\gamma - \alpha| = 2^{-n} |C - A| < \delta$, если $z \in \Delta_{nk}$,
    \begin{multline*}
        \left| \AnClOrInt[\AnClOrCurve[\partial \Delta_{nk}]] (f(z) - f(\alpha)) dz \right| \le \AnClOrInt[\AnClOrCurve[\partial \Delta_{nk}]] \left| f^*(M) - f^*(L) \right| dl(M) \le \\
        \le \AnClOrInt[\AnClOrCurve[\partial \Delta_{nk}]] \epsilon dl(M) = \epsilon (|\beta - \alpha| + |\gamma - \beta| + |\gamma - \alpha|) < 3 \epsilon |\gamma - \alpha| = \\
        = 3 \epsilon \cdot 2^{-n}|C - A|, \tag{17}
    \end{multline*}
    где $\alpha \in \C \leftrightarrow L \in \R^2$, $z \in \C \leftrightarrow M \in R^2$.
    Теперь $(15)$--$(17)$ при выбранном $n$ влечет:
    \[\left| \AnClOrInt[\AnClOrCurve[\partial \Delta]] f(z) dz \right| \le \sum_{k=1}^{2^n} \left| \AnClOrInt[\AnClOrCurve[\partial \Delta_{nk}]] f(z) dz \right| \le \sum_{k=1}^{2^n} 3 \epsilon \cdot 2^{-n}|C - A| = 3 \epsilon |C - A| \tag{18}\]
    Поскольку $\epsilon > 0$ произвольно, то $(18) \implies (12)$.
\end{longProof}

\section{Теорема Коши для произвольного треугольника}\marginpar{02.03.23}
% треугольник дельта!
\begin{theorem}
    Имеется область $G$ и имеется произвольный треугольник $\Delta$, $\Delta \subset G$, $f \in A(G)$, тогда
    \[\int_{\overrightarrow{\partial \Delta}} f(z) dz = 0\]
\end{theorem}
\begin{longProof}
    Пусть $\angle A$~--- больший из углов треугольника.
    И рассмотрим два случая:
    \begin{enumerate}
        \item Рассмотрим случай когда $BC$ параллельна оси $x$.
              \begin{center}
                  \def\svgwidth{0.45\linewidth}
                  \import{figures}{cauchy_theorem_random_triangle1.pdf_tex}
              \end{center}
              Проведем высоту $AD$ в треугольнике, $AD \perp CB$.
              Если угол $B$ или $C$ прямой, то это в точности предыдущая ситуация, у нас не так, поэтому считаем, что $D \in CB, D \neq C, D \neq B$.
              Будем обходить в положительном направлении.
              Тогда криволинейный интеграл второго рода
              \[\AnClOrInt[\AnClOrCurve[CBA]] f(z) dz = \AnClOrInt[\AnClOrCurve[CDA]] f(z) dz + \AnClOrInt[\AnClOrCurve[DBA]] f(z) dz = 0+ 0 = 0,\]
              т.к. каждый из них прямоугольный.

        \item Теперь ни одна из сторон не параллельна оси $x$.
              \begin{center}
                  \def\svgwidth{0.45\linewidth}
                  \import{figures}{cauchy_theorem_random_triangle2.pdf_tex}
              \end{center}
              Будем считать, что $A,B,C$ не только наименования вершин, но и комплексные числа.
              Выберем такое $\theta$, что $e^{i\theta}(B-C) = |B - C|$.
              И треугольник повернется так, что сторона $BC$ станет параллельна вещественной оси.
              Обозначим
              \begin{gather*}
                  G_\theta = \{\zeta: \zeta = e^{i \theta}z, z \in G\} \\
                  \Delta_\theta = \{\zeta: \zeta = e^{i \theta}z, z \in \Delta\}
              \end{gather*}
              Тогда $f_\theta = f(e^{- i \theta} \zeta)$, когда $\zeta \in G_\theta$ и $f_\theta \in A (G_\theta)$.
              И выполнены  соотношения
              \begin{gather*}
                  \AnClOrInt[\AnClOrCurve[\partial \Delta_\theta]] f_\theta(\zeta) d\zeta = 0 \tag{1}\\
                  \AnClOrInt[\AnClOrCurve[\partial \Delta]] f(z) dz = e^{-i \theta}\AnClOrInt[\AnClOrCurve[\partial \Delta_\theta]] f_\theta(\zeta) d\zeta \tag{2}\\
                  (1), (2) \implies \int_{\overrightarrow{\partial \Delta}} f(z) dz = 0
              \end{gather*}
    \end{enumerate}
\end{longProof}

\section{Теорема Коши для области, ограниченной многоугольниками}
\begin{definition}[Многоугольник]
    Многоугольник~--- замкнутая кусочно-гладкая кривая, все гладкие границы, которой являются отрезками, а так же множество внутренних точек, ограниченных данной кривой.
\end{definition}

\begin{theorem}[О триангуляции]
    Допустим, что есть область $G$, такая что $\partial G = \bigcup_{k=1}^m P_k$~--- многоугольники, тогда существует конечное множество треугольников $\Delta_j, 1 \le j \le N$, которые обладают свойствами:
    \begin{itemize}
        \item $\Delta_j \subset G$
        \item $\bigcup_{j=1}^N \Delta_j = G$
        \item если $j \neq k$, то
              \begin{itemize}
                  \item либо $\Delta_j \cap \Delta_k = \varnothing$
                  \item либо $\Delta_j \cap \Delta_k$~--- одна точка
                  \item либо $\Delta_j \cap \Delta_k$~--- сторона $\Delta_j$ и $\Delta_k$
              \end{itemize}
    \end{itemize}
    Такое множество называется триангуляцией области $G$.
\end{theorem}
\begin{proof}
    Доказывается индукцией по общему количеству вершин.
    Должно было быть в топологии.
\end{proof}
\begin{example}
    Пример триангуляции, где $P_1$~--- внешняя граница, а $P_2$~--- граница <<вырезанного>> куска.
    \begin{center}
        \def\svgwidth{0.4\linewidth}
        \import{figures}{triangulation.pdf_tex}
    \end{center}
\end{example}
\begin{definition}[Ориентация многоугольника]
    Пусть имеется область, граница которой состоит из более, чем одного многоугольника, тогда есть самый внешний многоугольник, внутри которого лежат остальные.
    Ориентированной границей всей области по определению будем называть задание какой-то ориентации на самом внешнем многоугольнике и задание противоположной на всех внутренних многоугольниках.

    Положительной ориентацией будем называть ориентацию, когда на внешнем многоугольнике задана положительная ориентация, а на всех внутренних задана отрицательная ориентация. В таком случае, если мы рассмотрим задание ориентации через вектор $\nu$, то все нужные точки будут лежать внутри области.
\end{definition}

\begin{proposition}
    Пусть имеется область $G$, граница которой $\AnClOrCurve[\partial G] = \overrightarrow{\bigcup_{k=1}^m P_k}$\footnote{В данном случае стрелочка над объединением, а так же запись вида $\overrightarrow{P_k}^\bullet$, означают задание ориентации в соответствии с определением выше.}.
    Имеется триангуляция этой области с помощью треугольников $\Delta_j$, и задана любая функция $f \in C(G)$.
    Тогда справедливо равенство
    \begin{gather*}
        \AnClOrInt[\AnClOrCurve[\partial G]] f(z) dz \coloneq \sum_{k=1}^{m} \AnClOrInt[\AnClOrCurve[P_k]^\bullet] f(z) dz \tag{3}
        \intertext{и}
        \AnClOrInt[\AnClOrCurve[\partial G]] f(z) dz = \sum_{j=1}^{N} \AnClOrInt[\AnClOrCurve[\partial \Delta_j]] f(z) dz \tag{4}
    \end{gather*}
\end{proposition}
\begin{longProof}
    Если $\partial\Delta_j = t_{j1} \cup t_{j2} \cup t_{j3}$, где $t_{jl}$~--- стороны треугольника $\Delta_j$.
    Тогда
    \[\sum_{j=1}^{N} \AnClOrInt[\AnClOrCurve[\partial \Delta_j]] f(z) dz = \sum_{j=1}^{N} \sum_{l=1}^{3} \AnClOrInt[\AnClOrCurve[t_{jl}]] f(z) dz\]
    Рассмотрим 3 случая:
    \begin{enumerate}
        \item Если $t_{jl}$ не лежит на $P_k$, $k=1,\dotsc, m$, то она обходится в одном из треугольников в положительном направлении, а в другом в отрицательном.
              Поэтому
              \[\sum_{j=1}^{N} \sum_{l=1}^{3} \AnClOrInt[\AnClOrCurve[t_{jl}]] f(z) dz = 0, \tag{5}\]
              где все $t_{jl}$ не лежат на $P_k$, таким образом в сумме (4) остаются только интегралы по границам, лежащим на $P_k$.
        \item Назовем $P_1$ внешний многоугольник.
              Если $t_{jl} \subset P_1$, тогда она имеет ту же положительную ориентацию, что и $P_1$.
              Раз триангуляция дает всю область $G$, то есть весь многоугольник $P_1$ будет покрыт этими непересекающимися отрезками.
        \item Назовем $P_2$ внутренний многоугольник.
              Аналогично, если $t_{jl} \subset P_2$, тогда она имеет ту же отрицательную ориентацию, что и $P_2$.
    \end{enumerate}

    При $t_{tl} \subset \bigcup P_k$ по свойству \ref{1:PartsSum}
    \begin{gather*}
        \sum_{j=1}^{N} \sum_{l=1}^{3} \AnClOrInt[\AnClOrCurve[t_{jl}]] f(z) dz = \sum_{j=1}^{N} \AnClOrInt[\AnClOrCurve[P_j]] f(z) dz \tag{6}
    \end{gather*}
    $(5)$, $(6) \implies (4)$.
\end{longProof}

\begin{theorem}
    Есть область $\Omega$, $f \in A (\Omega)$, $G \subset \Omega$, $\partial G = \bigcup_{k=1}^m P_k$, тогда
    \[\AnClOrInt[\AnClOrCurve[\partial G]] f(z) dz = 0 \tag{7}\]
\end{theorem}
\begin{proof}
    Заведем триангуляцию $G = \bigcup_{j=1}^N \Delta_j$ со всеми свойствами из теоремы.
    Тогда соотношение $(4)$ влечет:
    \begin{gather*}
        \AnClOrInt[\AnClOrCurve[\partial G]] f(z) dz = \sum_{j=1}^{N} \AnClOrInt[\AnClOrCurve[\partial \Delta_j]] f(z) dz = \sum_{j=1}^{N} 0 = 0
    \end{gather*}
\end{proof}

\section{Теорема Коши для областей, ограниченных кусочно-гладкими кривыми}
\begin{lemma}
    Есть кусочно-гладкая кривая $\gamma$, $a, b$~--- концы $\gamma$, т.ч. $a$~--- начало, $b$~--- конец, $f \in C(\gamma)$, тогда
    \[\left| \AnClOrInt[\AnClOrCurve[\gamma]] f(z) dz -  c(b-a) \right| \le \max_{z \in \gamma} |f(z) - c| l(\gamma), \tag{8}\]
    где $l(\gamma)$~--- длина $\gamma$.
\end{lemma}
\begin{proof}
    \begin{gather*}
        c(b-a) = \int_{\overrightarrow{\gamma}} c dz\\
        \begin{multlined}
            \left| \AnClOrInt[\AnClOrCurve[\gamma]] f(z) dz -  c(b-a) \right| = \left| \AnClOrInt[\AnClOrCurve[\gamma]] f(z) dz - \int_{\overrightarrow{\gamma}} c dz \right| = \left| \int_{\overrightarrow{\gamma}} (f(z) - c) dz \right| \le \\
            \le \int_{\gamma}\left|f(z) -c \right| dl(\gamma) \le \int_{\gamma}\max_{z \in \gamma} |f(z) - c| dl(\gamma) = \max_{z \in \gamma} |f(z) - c| l(\gamma)
        \end{multlined}
    \end{gather*}
\end{proof}
\begin{definition}[Ориентация области, ограниченной кусочно-гладкими кривыми]
    Имеется область $G$, т.ч. $\partial G = \bigcup_{k=1}^m \Gamma_k$~--- кусочно-гладкие кривые.
    Допустим, что $\Gamma_1$ содержит $\Gamma_k, k \ge 2$.
    Будем называть положительной ориентацией положительную ориентацию $\Gamma_1$ и отрицательную ориентацию остальных $\Gamma_k$.
\end{definition}
\begin{theorem}
    Есть область $\Omega$, $G \subset \Omega$~--- замкнутая область, $\partial G \subset \Omega$ и граница состоит из конечного числа кусочно-гладких кривых $\Gamma_k$, а ее внутренность $\mathring{G}$,$f \in A(\Omega)$. Тогда
    \[\AnClOrInt[\AnClOrCurve[\partial G]] f(z) dz = 0 \tag{9}\]
\end{theorem}
\begin{longProof}
    $G$~--- компакт, поэтому
    \begin{gather*}
        \exists \delta_0 > 0 \text{ т.ч. } \forall z \in G\quad \overline{B}_{\delta_0}(z) = \{\zeta : |\zeta - z| \le \delta_0\} \subset \Omega\\
        G_{\delta_0} = \bigcup_{z \in G} B_{\delta_0} (z)\text{ --- компакт}
    \end{gather*}
    Тогда $G_{\delta_0} \subset \Omega$ и $f \in C (G_{\delta_0})$.
    По теореме Кантора
    \begin{multline*}
        \forall \epsilon >0\ \exists \delta > 0 \text{ т.ч. }\forall \zeta_1, \zeta_2 \in G_{\delta_0} \text{ т.ч. } |\zeta_1 - \zeta_2| < \delta \\
        \implies |f(\zeta_1) - f(\zeta_2)| < \epsilon \tag{10}
    \end{multline*}
    Хотим через $P_{k,L} = \{z_{k,\nu}\}_{\nu = 1}^{N_{k,\delta}}$ обозначить разбиение $\overrightarrow{\Gamma_k}$, где $z_{k,\nu} \in \Gamma_k$ и
    \begin{gather*}
        |z_{k, \nu+1} - z_{k, \nu}| < \delta \tag{11}\\
        \left| \int_{\overrightarrow{\Gamma_k}} f(z) dz - \sum_{\nu = 1}^{N_{k,\delta}} f(z_{k,\nu}) (z_{k, \nu + 1} - z_{k, \nu}) \right| < \epsilon \tag{12} % по свойству 4
    \end{gather*}
    Теперь рассмотрим многоугольники $Q_{k, \delta}$ c вершинами $z_{k, \nu}, 1 \le \nu \le N_{k, \delta}$.
    Тогда, если мы ориентируем эти многоугольники, так чтобы общая граница была ориентирована положительно, то по предыдущей теореме
    \[\sum_{k=1}^{m} \int_{\overrightarrow{Q_{k, \delta}}^\bullet} f(z) dz = 0 \tag{13}\]
    Рассмотрим
    \begin{multline*}
        \left| \int_{\overrightarrow{Q_{k, \delta}}} f(z) dz  - \sum_{\nu = 1}^{N_{k, \delta}} f(z_{k, \nu}) (z_{k, \nu + 1} - z_{k, \nu})\right| = \\
        = \left| \sum_{\nu = 1}^{N_{k, \delta}} \int_{\mathrlap{\overrightarrow{[z_{k, \nu}, z_{k, \nu + 1}]}}}\  f(z) dz - \sum_{\nu = 1}^{N_{k, \delta}} f(z_{k, \nu}) (z_{k, \nu + 1} - z_{k, \nu}) \right| \le \\
        \le \sum_{\nu = 1}^{N_{k, \delta}} \left| \int_{\mathrlap{\overrightarrow{[z_{k, \nu}, z_{k, \nu + 1}]}}}\  f(z) dz -  f(z_{k, \nu}) (z_{k, \nu + 1} - z_{k, \nu}) \right| \le \\
        \le \sum_{\nu = 1}^{N_{k, \delta}} \max_{z \in [z_{k, \nu}, z_{k, \nu + 1}]} |f(z) - f(z_{k, \nu})| \cdot |z_{k, \nu + 1} - z_{k, \nu}| \tag{14}
    \end{multline*}
    В силу (10), а так же факта того, что сумма $|z_{k, \nu + 1} - z_{k, \nu}|$ не превосходит длины кривой, получаем \marginpar{09.03.23}
    \[(14) < \epsilon \sum_{\nu = 1}^{N_{k, \delta}} |z_{k, \nu + 1} - z_{k, \nu}| \le \epsilon l(\Gamma_k) \tag{15}\]
    \begin{multline*}
        \left| \int_{\overrightarrow{\partial G}} f(z) dz \right| \underset{(13)}{=}
        \left| \int_{\overrightarrow{\partial G}} f(z) dz  - \sum_{k=1}^{m} \int_{\overrightarrow{Q_{k, \delta}}^\bullet} f(z) dz  \right| \le \\
        \le \left| \int_{\overrightarrow{\partial G}} f(z) dz  - \sum_{k=1}^{m} \sum_{\nu = 1}^{N_{k, \delta}} f(z_{k,\nu}) (z_{k, \nu + 1} - z_{k, \nu})\right| +\\
        + \left| \sum_{k=1}^{m} \sum_{\nu = 1}^{N_{k, \delta}} f(z_{k,\nu}) (z_{k, \nu + 1} - z_{k, \nu}) - \sum_{k=1}^{m} \int_{\overrightarrow{Q_{k, \delta}}^\bullet} f(z) dz \right| < \\
        < m \epsilon + \epsilon \sum_{k=1}^{m} l(\Gamma_k) \tag{16}
    \end{multline*}
    Последнее неравенство следует из того, что в первом модуле интеграл по $\partial G$ есть сумма интегралов по $\Gamma_k$, тогда знак суммы можно вынести за модуль, и подмодульное выражение будет в точности неравенство (12).
    Во втором модуле сумма выносится аналогично, а дальше выполняется соотношение (15).
    В силу произвольности $\epsilon > 0$
    \[(16) \implies \int_{\overrightarrow{\partial G}} f(z) dz = 0\]
\end{longProof}
\end{document}
