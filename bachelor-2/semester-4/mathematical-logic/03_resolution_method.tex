% !TeX root = ./main.tex
\documentclass[main]{subfiles}
\begin{document}
\chapter[Метод резолюции]{Метод резолюции для исчисления высказываний}
\begin{definition}[Литерал]
    Литералом называется атомарная формула~(кроме $0$ и $1$), или её отрицание~(то есть либо $x$, либо $\lnot x$).
\end{definition}
В дальнейшем $l$ будет обозначать литерал.
\begin{definition}[Исчисление высказываний]
    \

    \begin{enumerate}
        \item Алфавит: конечное множество пропозициональных переменных
        \item Формулы: предложения~(клозы, clauses), то есть элементарные дизъюнкции~(дизъюнкции литералов)
        \item Правила вывода:
              \begin{itemize}
                  \item правило резолюции~(обозначения самого вывода не совпадают, оставил за собой право написать так, как считаю нужным):
                        \begin{gather*}
                            \begin{prooftree}
                                \hypo{D_1 \lor x \lor D_2}
                                \hypo{D_3\lor \lnot x\lor D_4}
                                \infer2{D_1\lor D_2 \lor D_3 \lor D_4}
                            \end{prooftree}
                        \end{gather*}
                  \item правило сокращения повторений:
                        \begin{gather*}
                            \begin{prooftree}
                                \hypo{D_1 \lor l\lor D_2\lor l\lor D_3}
                                \infer1{D_1\lor l\lor D_2\lor D_3}
                            \end{prooftree}
                        \end{gather*}
              \end{itemize}
        \item Аксиомы: нет
    \end{enumerate}
\end{definition}
\begin{definition}[Пустое предложение]
    $\nill\ \mid \ \varnothing; \ 0; \Lambda$.
\end{definition}
\begin{definition}[Неудовлетворительное множество предложений]
    Множество предложений называется неудовлетворительным, если из него выводимо пустое предложение.
\end{definition}
\begin{remark}
    Единственный случай, когда множество неудовлетворительное:
    \begin{gather*}
        \begin{prooftree}
            \hypo{x}
            \hypo{\lnot x}
            \infer2{\nill}
        \end{prooftree}
    \end{gather*}
\end{remark}
Встаёт вопрос, как доказывать какие-либо выражения в таком исчислении? Разберём общий случай. Хотим доказать
\[A_1 \dots A_n \Rightarrow B_1 \dots B_m \tag{*}\]
где $A_i$ $B_j$~--- это какие-то выражения, которые можно привести к виду элементарных дизъюнкций.

Алгоритм действий следующий:
\begin{enumerate}
    \item Раскрываем $(*)$:
          \begin{multline*}
              A_1 \and\dots\and A_n \to B_1\lor \dots \lor B_m  \lra\\
              \lra \lnot(A_1 \and\dots\and A_n)\lor B_1\lor \dots \lor B_m \lra\\
              \lra \lnot A_1 \lor \dots \lor \lnot A_n \lor B_1 \lor\dots\lor B_m
          \end{multline*}
    \item Если $(*)$ истинно, то $\lnot (*)$ должно быть ложным. Раскроем $\lnot(*)$:
          \begin{multline*}
              \lnot(\lnot A_1 \lor \dots \lor \lnot A_n \lor B_1 \lor\dots\lor B_m) \lra\\
              \lra A_1 \and \dots \and A_n \and \lnot B_1\and\dots \and \lnot B_m \tag{**}
          \end{multline*}
    \item Убрав конъюнкции, $(**)$ можно записать в виде:
          \begin{gather*}
              \begin{prooftree}
                  \hypo{A_1}
                  \hypo{\dots}
                  \hypo{A_n}
                  \hypo{\lnot B_1}
                  \hypo{\dots}
                  \hypo{\lnot B_m}
                  \infer6{\text{вывод из правила резолюции}}
              \end{prooftree}
          \end{gather*}
    \item Если мы смогли получить хотя бы один $\nill$ в выводе, то мы доказали, что $\lnot(*)$ ложно, а это значит, что $(*)$ истинно.
\end{enumerate}
В виду того, что лично я ничего не понял без \href{https://ru.wikipedia.org/wiki/Правило_резолюций#Метод_резолюции}{примера}, приведу его здесь:
\begin{example}
    Пусть в наших обозначениях будет:
    \begin{gather*}
        A_1 = x_1\and x_2 \\
        A_2 = x_1 \to x_3 \lra A_2 = \lnot x_1 \lor x_3\\
        B = x_3
    \end{gather*}
    Хотим доказать следующее: \[A_1 A_2 \Rightarrow B \tag{*}\]
    \begin{enumerate}
        \item $(*) \lra \lnot A_1 \lor \lnot A_2 \lor B$
        \item $\lnot (*) \lra A_1 \and A_2 \and \lnot B \quad (**)$
        \item Вспомним, что мы обозначали за $A_1, A_2, B$ и получим:
              \begin{gather*}
                  \begin{prooftree}
                      \hypo{x_1 \and x_2}
                      \hypo{\lnot x_1 \lor x_3}
                      \hypo{x_3}
                      \infer3{x_1\quad x_2\quad \lnot x_1 \lor x_3 \quad x_3}
                      \infer1[правило резолюции к $x_1$ и $\lnot x_1 \lor x_3$]{x_2 \quad \lnot x_3 \quad x_3}
                      \infer1[правило резолюции к $x_3$ и $\lnot x_3$]{x_2 \quad \nill}
                  \end{prooftree}
              \end{gather*}
        \item Получили $\nill$, значит, доказали, что $(*)$ верна
    \end{enumerate}
\end{example}
\end{document}