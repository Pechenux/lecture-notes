% !TeX root = ./main.tex
\documentclass[main]{subfiles}
\begin{document}
\chapter[СИВ]{Секвенциальное исчисление высказываний (СИВ)}\marginpar{21.02.23}
\section{Исчисления}
\begin{definition}[Исчисление]
    Для того, чтобы задать исчисление, достаточно задать:
    \begin{enumerate}
        \item Алфавит (конечный набор символов)
        \item Формулы исчисления (подмножество выражений из символов алфавита)
        \item Правила вывода (отношения между формулами\\
              $R(f_1, \dots ,f_n, f_{n+1})$ т.ч. если <<верен>> набор первых $n$ формул, то <<верна>> и последняя)
        \item Аксиомы (множество формул, которые <<верны>> в этом исчислении)
    \end{enumerate}
    Заметьте, что слово <<верно>> здесь взято в кавычки, потому что такого понятия изначально нет. Исчисление~--- это строго формальная вещь без семантики.
\end{definition}
\begin{definition}[Непосредственно выводимая формула]
    Формула $f$ называется непосредственно выводимой из множества формул $f_1, \dots, f_n$, если имеется правило вывода, такое что $R(f_1, \dots, f_n, f)$.
\end{definition}
\begin{definition}[Выводимая формула]
    Формула $f$ называется выводимой в исчислении $C$, если существует последовательность формул $f_1, \dots, f_n, f$, такая что $f_j \, \forall j$ является непосредственно выводимой из некоторых предыдущих или является аксиомой. Обозначение: $(\models_C f)$.
\end{definition}
\begin{definition}[Полнота]
    Исчисление называется полным, если всякая истинная формула выводима в этом исчислении.
\end{definition}
\begin{definition}[Непротиворечивость]
    Исчисление называется непротиворечивым, если не существует формулы $f$, такой что $\models f$ и $\models \lnot f$.
\end{definition}


\section{Секвенциальное исчисление высказываний}
\begin{definition}[Секвенция]
    Формулы СИВ~--- секвенция. То есть выражения вида $\Gamma \vdash \Delta$, где $\Gamma, \Delta$~--- наборы пропозициональных формул.
\end{definition}
\begin{definition}[Формульный образ секвенции]
    \[\Phi(\overbrace{A_1, \dots, A_n}^{\text{антецедент}}
        \vdash \overbrace{B_1, \dots , B_m}^{\text{сукцедент}}) \eqcirc A_1 \and A_2\and\dots \and A_n \to (B_1 \lor B_2 \lor \dots \lor B_m),\] где $\eqcirc$~--- графическое равенство, т.е. то, что слева это <<точь в точь>> то, что справа.
\end{definition}
\subsection{Правила вывода в СИВ}

$(*\ \vdash)$~--- правило вывода в антецеденте для бинарной связки $*$

$(\vdash \ *)$~--- правило вывода в сукцеденте для бинарной связки $*$

Далее будут выписаны (почти) все правила вывода:

\begin{align*}
     & \begin{prooftree}
           \hypo{\Gamma_1 A\Gamma_2 B\Gamma_3 &\vdash \Delta}
           \infer[left label=$(\and \ \vdash)$]1{\Gamma_1 A \and B \Gamma_2 \Gamma_3 &\vdash \Delta}
       \end{prooftree}       &
     & \begin{prooftree}
           \hypo{ \Gamma &\vdash \Delta_1 B\Delta_2}
           \infer[no rule]1{ \Gamma &\vdash \Delta_1A\Delta_2 }
           \infer[left label=$(\vdash \ \and)$]1{ \Gamma &\vdash \Delta_1 A \and B \Delta_2}
       \end{prooftree}
    \\
    \\
     & \begin{prooftree}
           \hypo{\Gamma_1 B\Gamma_2 &\vdash \Delta}
           \infer[left label=$(\lor \ \vdash)$]1{\Gamma_1 A\lor B\Gamma_2 &\vdash \Delta}
       \end{prooftree}                  &
     & \begin{prooftree}
           \hypo{\Gamma &\vdash \Delta_1A\Delta_2B\Delta_3}
           \infer[left label=$(\vdash \ \lor)$]1{\Gamma &\vdash \Delta_1 A\lor B\Delta_2\Delta_3}
       \end{prooftree}
    \\
    \\
     & \begin{prooftree}
           \hypo{\Gamma_1B\Gamma_2 &\vdash\Delta_1\Delta_2}
           \infer[no rule]1{\Gamma_1\Gamma_2 &\vdash \Delta_1 A \Delta_2}
           \infer[left label=$(\to \ \vdash)$]1{\Gamma_1 A\to B\Gamma_2 &\vdash \Delta_1 \Delta_2}
       \end{prooftree}         &
     & \begin{prooftree}
           \hypo{\Gamma_1A\Gamma_2 &\vdash \Delta_1B\Delta_2}
           \infer[left label=$(\vdash \ \to)$]1{\Gamma_1\Gamma_2 &\vdash \Delta_1A\to B \Delta_2}
       \end{prooftree}
    \\
    \\
     & \begin{prooftree}
           \hypo{\Gamma_1 \Gamma_2 &\vdash \Delta_1A\Delta_2}
           \infer[left label=$(\lnot\ \vdash)$]1{\Gamma_1 \lnot A\Gamma_2 &\vdash \Delta_1\Delta_2}
       \end{prooftree}        &
     & \begin{prooftree}
           \hypo{\Gamma_1 A\Gamma_2 &\vdash \Delta_1\Delta_2}
           \infer[left label=$(\vdash \ \lnot )$]1{\Gamma_1 \Gamma_2 &\vdash \Delta_1\lnot A\Delta_2}
       \end{prooftree}
    \\
    \\
     & \begin{prooftree}
           \hypo{\Gamma_1 \Gamma_2 &\vdash \Delta_1 A \Delta_2 B}
           \infer[no rule]1{\Gamma_1 A \Gamma_2 B &\vdash \Delta_1 \Delta_2}
           \infer[left label=$(\lra \ \vdash )$]1{\Gamma_1 A\lra B \Gamma_2 &\vdash \Delta_1 \Delta_2}
       \end{prooftree}     &
     & \begin{prooftree}
           \hypo{\Gamma_1 A\Gamma_2 &\vdash \Delta_1 B \Delta_2}
           \infer[no rule]1{\Gamma_1 B \Gamma_2 &\vdash \Delta_1 A \Delta_2}
           \infer[left label=$(\vdash \ \lra)$]1{\Gamma_1 \Gamma_2 &\vdash \Delta_1 A\lra B \Delta_2}
       \end{prooftree}
    \\
    \\
     & \begin{prooftree}
           \hypo{\Gamma_1 A \Gamma_2 &\vdash \Delta_1 B \Delta_2}
           \infer[no rule]1{\Gamma_1 B\Gamma_2  &\vdash \Delta_1 A\Delta_2}
           \infer[left label=$(\oplus \ \vdash )$]1{\Gamma_1 A\oplus B \Gamma_2 &\vdash \Delta_1 \Delta_2}
       \end{prooftree} &
     & \begin{prooftree}
           \hypo{\Gamma_1 \Gamma_2 &\vdash \Delta_1 AB \Delta_2}
           \infer[no rule]1{\Gamma_1 AB \Gamma_2 &\vdash \Delta_1  \Delta_2}
           \infer[left label=$(\vdash \ \oplus)$]1{\Gamma_1 \Gamma_2 &\vdash \Delta_1 A\oplus B \Delta_2}
       \end{prooftree}
\end{align*}
В дальнейшем будут использоваться обозначения $(*\ \vdash)$ и $(\vdash \ *)$, ссылаясь на эту таблицу.

\begin{definition}[Секвенциальное исчисление высказываний]
    \

    \begin{enumerate}
        \item Алфавит:\begin{itemize}
                  \item символы для обозначения пропозициональных переменных (например, $(x, y, z, 0, \dots, 9)$)
                  \item логические связки $\and, \lnot$
                  \item знак секвенции $\vdash$
              \end{itemize}
        \item Формулы: секвенции из пропозициональных формул
        \item Правила вывода:\begin{itemize}
                  \item $(\vdash \ \and)$
                  \item $(\and \ \vdash)$
                  \item $(\vdash \ \lnot)$
                  \item $(\lnot \ \vdash)$
              \end{itemize}
        \item Аксиомы: любые формулы вида $\Gamma_1 A \Gamma_2 \vdash \Delta_1 A \Delta_2$
    \end{enumerate}
\end{definition}
\begin{lemma}
    Формульный образ аксиомы является тавтологией.
\end{lemma}
\begin{proof}
    Если применить правило де~Моргана и раскрыть импликацию, то аргумент $\Phi$ в общем виде можно записать так:
    \begin{gather*}
        A_1 \and \ldots \and A_n \to B_1 \lor \dots \lor B_m \lra \lnot A_1 \lor \dots \lor \lnot A_n \lor B_1 \lor \dots \lor B_m\\
        \intertext{из этого следует, что }
        \Phi(A_1, \dots , A_k, A, A_{k+1}, \dots, A_n \vdash B_1, \dots, B_l, A, B_{l+1}, \dots, B_m) \lra \\
        \lra \lnot A_1 \lor \dots \lor \lnot A_k \lor \textcolor{red}{\lnot A} \lor \lnot A_{k+1}\lor\dots \lor\lnot A_n \lor\\
        \lor B_1\lor\dots \lor B_l \lor \textcolor{red}{A} \lor B_{l+1}\lor \dots \lor B_m
    \end{gather*}
    \textcolor{red}{красное} даёт нам тавтологию.
\end{proof}

\begin{definition}[Допустимое правило]
    Правило вывода называется допустимым, если по всякому выводу, содержащему применение этого правила можно построить вывод без применения этого правила.
\end{definition}
\begin{proposition}
    В СИВ допустимыми являются:
    \begin{gather*}
        (\lor \ \vdash), (\vdash \ \lor), (\to \ \vdash), (\vdash \ \to), (\lra \ \vdash), (\vdash \ \lra), (\oplus \ \vdash), (\vdash\ \oplus),\\
        (\mid \ \vdash), (\vdash \ \mid), (\downarrow \ \vdash), (\vdash \ \downarrow).
    \end{gather*}
\end{proposition}
\begin{proof}
    Здесь будет приведено только доказательство допустимости $(\lor \ \vdash)$, остальное доказывается аналогично.
    \begin{gather*}
        \begin{prooftree}
            \hypo{\Gamma_1 B \Gamma_2 &\vdash \Delta \quad s_1 }
            \infer[no rule]1{\Gamma_1 A \Gamma_2 &\vdash \Delta \quad s_2}
            \infer[left label=$(\lor \ \vdash)$]1{\Gamma_1 \lnot (\lnot A \and \lnot B) \Gamma_2 &\vdash \Delta \quad s}
        \end{prooftree}
    \end{gather*}
    $s$ может быть получено из $s'$ по $(\lnot \ \vdash)$, где

    \[s' : \quad \Gamma_1 \Gamma_2 \vdash \Delta (\lnot A \and \lnot B)\]

    $s'$ может быть получено из $s''$ и $s'''$ по $(\vdash \ \and)$, где
    \begin{gather*}
        s'': \quad \Gamma_1\Gamma_2 \vdash \Delta \lnot A\\
        s''' : \quad \Gamma_1\Gamma_2 \vdash \Delta \lnot B
    \end{gather*}
    $s''$ может быть получена из $s_2$ по $(\lnot \ \vdash)$

    $s'''$ может быть получена из $s_1$ по $(\lnot \ \vdash)$.
\end{proof}

\begin{align*}
    \intertext{Правила сокращения повторений: }
     & \begin{prooftree}
           \hypo{\Gamma_1 A\Gamma_2 A\Gamma_3 &\vdash \Delta}
           \infer1{\Gamma_1 A\Gamma_2 \Gamma_3 &\vdash \Delta}
       \end{prooftree}    &
     & \begin{prooftree}
           \hypo{ \Gamma &\vdash \Delta_1 A \Delta_2 A \Delta_3}
           \infer1{\Gamma &\vdash \Delta_1 A \Delta_2 \Delta_3}
       \end{prooftree}
    \\
    \intertext{Правила перестановки:}
     & \begin{prooftree}
           \hypo{\Gamma_1 A \Gamma_2 B \Gamma_3 &\vdash \Delta}
           \infer1{\Gamma_1 B \Gamma_2 A \Gamma_3 &\vdash \Delta}
       \end{prooftree} &
     & \begin{prooftree}
           \hypo{\Gamma &\vdash \Delta_1 A \Delta_2 B \Delta_3}
           \infer1{\Gamma &\vdash \Delta_1 B \Delta_2 A \Delta_3}
       \end{prooftree}
    \\
    \intertext{Правила добавления:}
     & \begin{prooftree}
           \hypo{\Gamma_1 \Gamma_2 &\vdash \Delta}
           \infer1{\Gamma_1 A\Gamma_2 &\vdash \Delta}
       \end{prooftree}             &
     & \begin{prooftree}
           \hypo{\Gamma &\vdash \Delta_1 \Delta_2}
           \infer1{\Gamma &\vdash \Delta_1 A \Delta_2}
       \end{prooftree}
\end{align*}
Каждое из этих правил является очевидным, если расписать формульные образы.
\begin{proposition}
    В СИВ допустимы обратные правила:
    \begin{gather*}
        \begin{prooftree}
            \hypo{\Gamma_1 A \and B \Gamma_2 &\vdash \Delta}
            \infer[left label=$(\and^- \ \vdash)$]1{\Gamma_1 A B \Gamma_2 &\vdash \Delta}
        \end{prooftree}
        \\
        \begin{prooftree}
            \hypo{\Gamma &\vdash \Delta_1 A\and B \Delta_2}
            \infer[left label=$(\vdash \ \and^-_1)$]1{\Gamma &\vdash \Delta_1 A \Delta_2}
        \end{prooftree}
        \\
        \begin{prooftree}
            \hypo{\Gamma &\vdash \Delta_1 A\and B \Delta_2}
            \infer[left label=$(\vdash \ \and^-_2)$]1{\Gamma &\vdash \Delta_1 B \Delta_2}
        \end{prooftree}
    \end{gather*}
\end{proposition}
\begin{definition}[Производное правило]
    Правило вывода называется производным правила исчисления, если всякая формула, выводимая с применением этого правила является выводимой и без его применения.
\end{definition}
\begin{example}
    Правило сечения:
    \begin{gather*}
        \begin{prooftree}
            \hypo{\Gamma_3 \Gamma_4 &\vdash \Delta_3 A \Delta_4}
            \infer[no rule]1{\Gamma_1 A \Gamma_2 &\vdash \Delta_1 B \Delta_2}
            \infer1{\Gamma_1 \Gamma_2 \Gamma_3 \Gamma_4 &\vdash \Delta_1 \Delta_2 B \Delta_3 \Delta_4}
        \end{prooftree}
        \quad \rightsquigarrow\quad
        \begin{prooftree}
            \hypo{&\vdash A}
            \infer[no rule]1{A &\vdash B}
            \infer1{&\vdash B}
        \end{prooftree}
    \end{gather*}
\end{example}
\begin{lemma}
    Для всякого правила вывода в СИВ формульный образ заключения равносилен конъюнкции формул образов посылок.
\end{lemma}
\begin{proof}
    На примере $(\and\ \vdash)$ и $(\vdash \ \and)$.

    $(\and \ \vdash)$:

    \begin{description}
        \item[заключение: ] $\Phi(\Gamma_1 A\and B \vdash \Delta) \lra \lnot((\and\Gamma_1)\and A\and B\and (\and \Gamma_2))\lor (\lor\Delta)$
        \item[посылка: ] $\Phi(\Gamma_1 A B \Gamma_2 \vdash \Delta) \lra \lnot ((\and\Gamma_1)\and A\and B\and (\and \Gamma_2))\lor (\lor\Delta)$
    \end{description}
    $(\vdash \ \and)$:
    \begin{description}
        \item[посылка1: ] $\Phi(\Gamma \vdash \Delta_1 A \Delta_2)\lra \lnot((\and \Gamma)) \lor (\lor \Delta_1)\lor A\lor(\lor\Delta_2)$
        \item[посылка2: ] $\Phi(\Gamma \vdash \Delta_1 B \Delta_2)\lra \\ \lra \textcolor{red}{\lnot((\and \Gamma)) \lor (\lor \Delta_1)}\lor B\lor\textcolor{teal}{(\lor\Delta_2)}$
    \end{description}
    Обозначим \textcolor{red}{красное} за $F_1$, а \textcolor{teal}{зелёное} за $F_2$
    \begin{description}
        \item[посылка1 $\and$ посылка2: ] $(F_1 \lor A \lor F_2) \and (F_1\lor B\lor F_2) \lra \\ \lra (F_1 \lor F_2) \lor(A \and B)$
        \item[заключение: ] $\Phi(\Gamma \vdash \Delta_1 A\and B \Delta_2) \lra \\ \lra \lnot((\and \Gamma))\lor(\lor\Delta_1)\lor A\and B \lor(\lor \Delta_2)$
    \end{description}
\end{proof}
\begin{theorem}
    СИВ полно~(то есть формульный образ всякой секвенции, выводимой в СИВ является тавтологией).
\end{theorem}
\begin{proof}
    Как-то методом возвратной математической индукции.
\end{proof}
\begin{theorem}
    СИВ непротиворечиво.
\end{theorem}
\end{document}