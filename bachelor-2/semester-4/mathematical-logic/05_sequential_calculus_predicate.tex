% !TeX root = ./main.tex
\documentclass[main]{subfiles}
\begin{document}
\chapter[СИП]{Секвенциальное исчисление предикатов}
Обозначим символом $[P]_y^x$ результат замены всех секвенциальных переменных $x$  на переменную $y$; а символом $[P]_t^x$ результат замены всех секвенциальных переменных $x$  на терм $t$. В дальнейшем будем активно использовать эти обозначения.

Кванторные правила:
\begin{align*}
    \intertext{Следующие две формулы работают только если терм $t$ свободен для подстановки в формулу $P$ вместо переменной $x$}
     & \begin{prooftree}
           \hypo{\Gamma & \vdash \Delta_1 [P]_t^x \Delta_2}
           \infer[left label=$(\vdash \ \exists)$]1{\Gamma &\vdash \Delta_1 \exists x P \Delta_2}
       \end{prooftree}   &
     & \begin{prooftree}
           \hypo{\Gamma_1 [P]_t^x \Gamma_2 & \vdash \Delta}
           \infer[left label=$(\forall \ \vdash)$]1{\Gamma_1 \forall xP \Gamma_2 &\vdash \Delta}
       \end{prooftree}
    \\
    \intertext{Следующие две формулы работают только если переменная $y$ свободна для подстановки в формулу $P$ вместо переменной $x$ И НЕ ВХОДИТ СВОБОДНО В ЗАКЛЮЧЕНИЕ ПРАВИЛА!!!}
     & \begin{prooftree}
           \hypo{\Gamma &\vdash \Delta_1 [P]_y^x\Delta_2}
           \infer[left label=$(\vdash \ \forall)$]1{\Gamma & \vdash  \Delta_1 \forall x P \Delta_2}
       \end{prooftree} &
     & \begin{prooftree}
           \hypo{\Gamma_1 [P]_y^x \Gamma_2 &\vdash \Delta}
           \infer[left label=$(\exists \ \vdash)$]1{\Gamma_1 \exists x P \Gamma_2 & \vdash \Delta}
       \end{prooftree}
\end{align*}

\begin{definition}[Секвенциальное исчисление предикатов]
    \

    \begin{enumerate}
        \item Алфавит: \\
              $\{\text{алфавит для записи пропозициональных формул}, \vdash\}$
        \item Формулы: секвенции вида $\Gamma \vdash \Delta$,  где $\Gamma, \Delta$~--- списки пропозициональных формул
        \item Правила вывода: все правила вывода для СИВ и ещё кванторные правила:
              \begin{itemize}
                  \item $(\vdash \ \exists)$
                  \item $(\forall \ \vdash)$
                  \item $(\vdash \ \forall)$
                  \item $(\exists \ \vdash)$
              \end{itemize}
        \item Аксиомы: секвенции вида $\Gamma_1 A \Gamma_2 \vdash \Delta_1 A \Delta_2$
    \end{enumerate}
\end{definition}
Правило сокращения повторений:
\begin{gather*}
    \begin{prooftree}
        \hypo{\Gamma_1 P \Gamma_2 P \Gamma_3 & \vdash \Delta}
        \infer1{\Gamma_1 P \Gamma_2 \Gamma_3 & \vdash \Delta}
    \end{prooftree} \qquad
    \begin{prooftree}
        \hypo{\Gamma & \vdash \Delta_1 P \Delta_2 P \Delta_3}
        \infer1{\Gamma &\vdash \Delta_1 P \Delta_2 \Delta_3}
    \end{prooftree}
\end{gather*}
\begin{theorem}
    СИП полно и непротиворечиво.
\end{theorem}
\begin{proof}
    В курсе не будет.
\end{proof}
\end{document}
